\documentclass[uplatex,dvipdfmx,a4j]{jsreport}
\usepackage{thesis}

\begin{document}
  概要

  プログラミング言語において文字列は基本的なデータ型の一つであり文字列変数が多用される.
  文字列の操作はその正当性を決めることが難しく, 意図しない動作を含むことがある.
  文字列操作を解析する方法の一つに文字列制約を利用したものがあり, いくつかのソルバがある.
  既存のソルバはオートマトン, トランスデューサーを拡張したモデルを用いて充足可能性判定を行うが,
  問題点の一つとして文字列のドメインが大きくなった時指数的に実行時間が大きくなることがある.
  Symbolicなモデルはアルファベットの大きい場合にオートマトンの遷移で全ての文字に対する
  個別の遷移が必要になってしまう問題を解決するために提案されたものである.
  本論文ではSSTをSymbolicに拡張し, それによるオートマトンの逆像の構成を行った.
  また, その手法を用いて実際に文字列制約の充足可能性判定を行い, Rustによる実装を行った.
\end{document}
