\documentclass[uplatex,dvipdfmx,a4j]{jsreport}
\usepackage{thesis}

\begin{document}
  \chapter{Symbolic SST}  \label{chap:symbolic_sst}
  本研究では後に説明するように文字列制約をAlurらによるStreaming String TransducerをSymbolicに
  拡張したものとSymbolic Automatonを用いて表し, AutomatonのSSTによる逆像を取ることで充足可能性判定
  を行う. 以下のようにSymbolic SSTを定義する.

  \begin{definition}[Symbolic SST]

    Symbolic SST$S$を六組$\lrangle{\mathcal{A}, X, Q, q_0, f, \Delta}$で定義する.
    \begin{itemize}
      \item $\mathcal{A}$: Boolean Algebra
      \item $X$: 文字列変数
      \item $Q$: 状態
      \item $q_0$: 初期状態
      \item $F \subseteq Q \rightarrow (\mathcal{D} \cup X)^*$: 受理関数
      \item $\Delta \subseteq Q \times \Psi \times
      Q \times (X \rightarrow (\Lambda \cup X)^*)$: 遷移関係
    \end{itemize}
    出力される関数$\alpha \subseteq X \rightarrow (\Lambda \cup X)^*$
    を, function term$\lambda$に対しては$\alpha(\lambda) = \lambda$となるよう自然に拡張された関数
    $\alpha \subseteq (\Lambda \cup X)^* \rightarrow (\Lambda \cup X)^*$
    として考える.

    $\Lambda_k \defeq \Lambda \times k, \Lambda \simeq \Lambda \times 1$とする.
    演算$\uparrow$を$(\lambda, i)^\uparrow = (\lambda, i + 1)$と定義し,
    $\forall x \in X.\ x^\uparrow = x$として
    $\uparrow \subseteq
    (\Lambda_{[k]} \cup X)^* \rightarrow (\Lambda_{[k + 1]} \cup X)^*$
    に拡張する.
    ここで$\Lambda_{[k]} = \bigcup_{i \in \{ 1, \ldots, k \}} \Lambda_i$.

    同様に$\forall d \in \mathcal{D}.\ d^\uparrow = d$により
    $(X \cup \mathcal{D})^* \rightarrow (X \cup \mathcal{D})^*$に対しても拡張する.
    この関数によってSymbolic SSTの関数適用を
    $\forall h \in (\Lambda^{[k]} \times X)^*.\ \alpha(h) = \alpha(h^\uparrow)$
    と意味付ける.
    $\alpha = \alpha_1 \circ \cdots \circ \alpha_n$
    は$w = a_1 \cdots a_n$に対して
    $\alpha(z)$中の$(\lambda, i)$の出現を$\lambda(a_i)$で置換することで
    $\alpha_w \in X \rightarrow (\mathcal{D} \cup X)^*$に簡約される.

    $\Psi_{[k]} \in \Psi^{\{ 1, \ldots, k \}}$として,
    $\psi \in \Psi_{[k]}$に対して$\psi_i = \psi(i)$と書く事にする.

    $\phi \in \Psi_{[n]}$は長さ$n$の文字列$w = a_1 \ldots a_n$
    が$\bigwedge_{[n]} \phi_i(a_i)$を
    真にするとき$w \in \seman{\phi}$と意味付ける.
    $\Psi^{[k]}$に対して$\land, \lor, \lnot$を$i \in \{ 1, \ldots, k \}$
    \begin{align*}
      \varphi \land \psi (i) &= \varphi(i) \land \psi(i) \\
      \varphi \lor \psi (i) &= \varphi(i) \lor \psi(i) \\
      \lnot \psi (i) &= \lnot \psi(i)
    \end{align*}
    と定義し, $\Rightarrow$を
    $\varphi \Rightarrow \psi = (\lnot \varphi \lor \psi)$とする.
    また, $\uparrow \subseteq \Psi^{[k]} \rightarrow \Psi^{[k + 1]}$を
    \[
      \varphi^\uparrow(i) =
        \begin{cases}
          \top \quad &\mbox{if} \quad i = 1 \\
          \varphi(i - 1) \quad &\mbox{otherwise}
        \end{cases}
    \]
    と定義する.

    略記方として特に$\phi(i) = \varphi$かつ$j (\neq i)$に対して$\phi(j) = \top$
    であるような$\phi \in \Psi^{[k]}$を$[\varphi]_i$と書き, そのような論理式の集合を$\Psi_k$とする.

    $\Delta$は次のような関数である.
    \[
      q \transarrow{\varphi / \alpha}{} q'
    \]
    ここで$\alpha$は
    \begin{align*}
      x &= \kappa_{1, x} \kappa_{2, x} \ldots \kappa_{n_x, x} \\
      y &= \kappa_{1, y} \kappa_{2, y} \ldots \kappa_{n_y, y} \\
      \vdots &
    \end{align*}
    それぞれの$\kappa_{i, x}$はfunction term または文字列変数である.

    $a \in \mathcal{D}$と$S$の遷移に対して, $q \transarrow{\varphi / \alpha}{} q'$
    かつ$a \in \seman{\varphi}$であるとき, $q \transarrow{a / \alpha}{} q'$とする.

    再帰的に, $q_0, q_1, \ldots, q_n, w = a_1 a_2 \ldots a_n$に対して,
    $q_0 \transarrow{\varphi_1 / \alpha_1}{} q_1 \transarrow{\varphi_2 / \alpha_2}{}
    \cdots \transarrow{\varphi_n / \alpha_n}{} q_n$
    かつ
    $w \in \seman{\varphi_1 \land \cdots \land \varphi_n^{(\uparrow)^{n - 1}}}$
    であるとき
    $q_0 \transarrow{w / \alpha_w}{} q_n$とする.

    $S$の言語を
    $\mathcal{L}(S) =
    \{ (w, w') \in (\mathcal{D}^*)^2 \mid
    q_0 \transarrow{w / \alpha_w}{} q \land
    q \in \mbox{dom}(F) \land
    \hat{\varepsilon}(\alpha_w(F(q))) = w' \}$とする.

    ここで$\hat{\varepsilon}$は文字列変数を$\varepsilon$で置き換える関数である.
  \end{definition}

  \begin{definition}[Bounded Copy]
    更新関数$\alpha$と$x \in X$に対して$Var(\alpha(x))$を$\alpha(x)$に含まれる変数の数とする.

    Symbolic SST$S = \lrangle{\mathcal{A}, X, Q, q_0, f, \Delta}$が
    \[
      \underset{w \in \mathcal{A}^*}{\mbox{max}}
        \underset{x \in X, q_0 \transarrow{w / \alpha_w}{} q}{\mbox{max}} Var(\alpha(x)) = k
    \]
    を満たすとき$k$-bounded copyであるという. 特に$k = 1$であるときcopylessという.
  \end{definition}

  \begin{example}
    以下のSymbolic SSTは任意の文字列$w$に対して$ww^R$を出力する.
    ただし, $w^R$は$w$を反転させた文字列とする.
    \begin{center}
      \begin{tikzpicture}[aut]
        %nodes
        \node[state, initial, accepting] at (0, 0) (p) {$p$};
        %output function
        \node[below = 1 of p] {$F(p) = x \cdot y$};
        %arrows
        \draw [->]
          (p) edge[loop above] node{
          $\top /
          \left[
            \begin{aligned}
              x &= x \cdot \id \\
              y &= \id \cdot y
            \end{aligned}
          \right]$
          } (p)
        ;
      \end{tikzpicture}
    \end{center}
    このSymbolic SSTはcopylessである.
  \end{example}

  \begin{example}
    以下のSymbolic SSTは入力の, $"(a \cup b)^*c"$で表されるマッチを$"d"$で置き換えた文字列を
    出力する.
    $\varphi_{a,b}(\sigma) = (\sigma = a \lor \sigma = b)$,
    $\varphi_c(\sigma) = (\sigma = d)$,
      \begin{align*}
        f &= \left[
          \begin{aligned}
            x &= x \cdot \id  \\
            y &= x \cdot \id
          \end{aligned}
        \right] \\
        f' &= \left[
          \begin{aligned}
            x &= x \cdot \id  \\
            y &= y
          \end{aligned}
        \right] \\
        g &= \left[
          \begin{aligned}
            x &= y \cdot "d" \cdot  \id \\
            y &= y \cdot "d"
          \end{aligned}
        \right] \\
        g' &= \left[
          \begin{aligned}
            x &= y \cdot "d" \cdot  \id \\
            y &= y \cdot "d" \cdot  \id
          \end{aligned}
        \right]
      \end{align*}
    とする.

    出力関数は
    \begin{align*}
      F(p) &= x \\
      F(q) &= x \\
      F(r) &= y \cdot "d"
    \end{align*}
    である.
    \begin{center}
      \begin{tikzpicture}[aut]
        %nodes
        \node[state, initial, accepting] at (-3, 0) (p) {$p$};
        \node[state, accepting] at (3, 5) (q) {$q$};
        \node[state, accepting] at (3, -5) (r) {$r$};
        %arrows
        \draw [->]
          (p) edge[loop above] node[left=0.1] {$\lnot \varphi_{a,b} / f$} ()
              edge[bend left] node[above=0.5] {$\varphi_{a,b} / f'$} (q)
          (q) edge[loop above] node [above] {$\varphi_{a,b} / f'$} ()
              edge[bend left] node[left=0.2] {$\lnot (\varphi_{a,b} \land \varphi_{c}) / f$} (p)
              edge[bend right=13] node[left=0.3] {$\varphi_{c} / f'$} (r)
          (r) edge[bend right] node[below right] {$\varphi_{a,b} / g$} (q)
              edge[bend left] node[below left=0.5] {$\lnot \varphi_{a,b} / g'$} (p)
        ;
      \end{tikzpicture}
    \end{center}
    上記Symbolic SSTはcopylessである.
  \end{example}
\end{document}
