\documentclass[uplatex,dvipdfmx,a4j]{jsreport}
\usepackage{thesis}

\begin{document}
  \chapter{Symbolic SSTのSymbolic Automatonによる逆像}  \label{chap:symbolic_preimage}

  Symbolic SST $S = \lrangle{\mathcal{A}, X, Q, q_0, F_S, \Delta}$,
  SFA $A = \lrangle{\mathcal{A}, P, p_0, F, \delta}$に対して,
  \[
    \mathcal{L}(A') = S^{-1}(\mathcal{L}(A))
  \]
  を満たすようなSFA $A'$を構成する.
  すなわち, 入力に対するSSTの出力でSFAが受理するか否かを判定するSFAを構成する.

  基本的なアイデアとしてはSSTの入力に対し, 各ステップ毎の変数の値でSFAの各状態からどのように遷移しうるかを覚えておく
  ことで, SSTの各状態に対する出力で実際にSFAを動かし, 変数部分の遷移を覚えておいた遷移先で置換することにより
  最終的な入力の受理・拒否を判定する.

  このとき, Symbolicな遷移ではたとえ決定性のSST, SFAであったとしてもfunction termの適用結果が
  ある論理式を満たすかどうかは入力に依存してしまうため, 決定性のSSTを生成するためには全ての状態のシミュレーションを行う
  必要があり, 論理式のどの部分が実際に必要な条件か判定しなければならない.
  本研究では最終的な出力に必要とされる変数の条件のみを非決定的に選び, 論理式として採用するため, 最終的に
  構成されるSFAは非決定的になる.

  また上記のような構成を行うために, 逆像を表すSFAは初期状態としていくつかの状態を持ちうる.
  直接構成されるSFAに関しては\ref{chap:preliminalies}章の定義と一致しないが,
  新しい状態を用意し, 全ての開始状態の遷移をその状態に加え, 新しい開始状態とすることで前述の定義の形に
  することは容易であり本質的には無関係なため逆像を表すSFAでは開始状態の複数あるSFAを構成する.

  \section{構成法}
  逆像を表すSFA $A' = \lrangle{\mathcal{A}, Q', I, F', \delta'}$とする.
  $Q' = Q \times (X \rightarrow \mathcal{P}(P \times P))$,
  $F' \subseteq Q'$, $\delta' \subseteq Q' \times \Psi \rightarrow Q'$,
  $I = \{ (q_0, \id') \mid \forall x \in \dom(\id').\ \id'(x) \subseteq ID\
  s.t.\ ID = \{ (p, p) \mid p \in P \} \}$である.

  ここで状態$Q'$はSSTの状態$Q$と各変数に対して各状態からの遷移先を非決定的に表す集合を返す関数の組である.
  この関数が後段のSFAでの遷移をシミュレーションすることになる.
  $(q, g) \in Q'$はSSTの状態と変数による遷移を加えたSFAの組と見ることができる.
  この関数で各変数に対応する集合に含まれる状態の組もSSTの出力で動作するために必要なものに限られることに注意する. \\
  各変数の初期の値が$\varepsilon$であることから必要な変数のみを定義域とする関数で各状態から遷移しない関数であれば全て
  初期状態になりうるため, $I$はそれらとSSTの初期状態の組になっている.

  $G = X \rightarrow \mathcal{P}(P \times P)$とする.
  $g \in G$に対して
  関係$\transarrow{}{g} \subseteq (\Lambda_{k} \cup X) \times P \times \Psi_{[m]} \times P$を
  $p \transarrow{\varphi}{A} p'$のとき
  $p \transarrow{(\lambda, i) / [\varphi(\lambda)]_i}{g} p'$,
  $(p, p') \in g(x)$のとき$p \transarrow{x / \top}{} p'$として定義する.
  また,
  \[
    p \transarrow{h / \varphi}{g} p' \in (X \rightarrow \mathcal{P}(P \times P))
      \rightarrow (\bigcup_{m \in \mathcal{N}}
      ((\Lambda_{[m]} \cup X)^*) \times P \times \Psi_{[m]} \times P)
  \]
  を以下のように帰納的に拡張する.

  \setcounter{equation}{0}
  \begin{align}
    p &\transarrow{
      (\lambda, i) h / [\varphi(\lambda)]_i \land \varphi'
    }{g} p' \quad
      \mbox{iff} \
        p \transarrow{\varphi}{A} p'' \
      \mbox{and} \
        p'' \transarrow{h / \varphi'}{g} p' \
      \mbox{and} \
        (\lambda, i) \in \Lambda^{[k]}
      \mbox{and} \
        k \in [m]   \tag{\mbox{rule 1}} \label{rule: 1} \\
    p &\transarrow{x h / \varphi}{g} p' \quad
      \mbox{iff} \
        (p, p'') \in g(x) \land p'' \transarrow{h / \varphi}{g} p' \
      \mbox{and} \
        x \in X \tag{\mbox{rule 2}} \label{rule: 2} \\
    p &\transarrow{\varepsilon / [\top]}{g} p \tag{\mbox{rule 3}} \label{rule: 3}
  \end{align}

  記法として, $p, p' \in P, h \in (\Lambda \cup X)^*$に対して$\Delta_g(p, h, p') = \bigvee \{
    \varphi \mid p \transarrow{h / \varphi}{g} p'
  \}$と定義する. ただし, $\bigvee \emptyset = \bot$.
  また, $\transarrow{}{} \subseteq \bigcup_{m \in \mathcal{N}} G \times
    (X \rightarrow (X \cup \Lambda^{[m]})^*) \times (G \times \Psi_{[m]})$を
  $g, g' \in G$に対して,
  $\dom(g) = \bigcup_{y \in \scriptsize \dom(g')} Var(\alpha(y))$
  のとき
  $\phi =
    \bigwedge_{x \in \dom(g')} \bigwedge_{(p, p') \in g'(x)}
      \Delta_g(p, \alpha(x), p')
  $として
  \[
    g \transarrow{\alpha / \phi}{} g'
  \]
  と定義する.
  全ての論理式の和をとり$g$上の遷移を定義しても, 非決定的に全ての論理式に対して関係を付けても,
  必要な変数に関する条件のみを定義しておりこの後の議論はほとんど同様の方法で成り立つ. 本研究では論理式の和を取ることにする.

  $q \transarrow{\psi / \alpha}{} q' \in \Delta$かつ$g \transarrow{\alpha / \phi}{} g'$であるとき
  $(q, g) \transarrow{\phi \land \psi}{} (q', g') \in \delta'$が成り立つように$\delta'$を構成する.

  関係$p \transarrow{w}{g} p'\in (X \hookrightarrow P \times P)
  \rightarrow (\mathcal{D} \cup X)^* \times \mathcal{P}(P \times P)$
  を以下で帰納的に定義する.
  \setcounter{equation}{0}
  \begin{align}
    p &\transarrow{aw}{g} p' \
      \mbox{iff} \
        p \transarrow{a}{A} p'' \
      \mbox{and} \
        p'' \transarrow{w}{g} p' \
      \mbox{and} \
        a \in \mathcal{D} \tag{\mbox{rule 4}} \label{rule: 4} \\
    p &\transarrow{xw}{g} p' \
      \mbox{iff} \
        (p, p'') \in g(x) \land p'' \transarrow{w / \varphi}{g} p' \
      \mbox{and} \
        x \in X \tag{\mbox{rule 5}} \label{rule: 5} \\
    p &\transarrow{\varepsilon}{g} p \tag{\mbox{rule 6}} \label{rule: 6}
  \end{align}
  と定義する.

  関係$\transarrow{}{g}$は最終的に構成するSFAの意味論には無関係だが, 変数によるSFAの遷移を表すために必要となる.

  $F' = \{ (q, g) \in Q' \mid q \in \dom(F_S) \land
  p_0 \transarrow{F_S(q)}{g} p_f, p_f \in F \}$を満たすように構成する.
  $F'$はSSTで出力を持つ状態とSFAでの遷移が受理状態となるような状態の集合となる.

  \section{構成例}
  SFA $A$をSymbolic SST $S$で逆像を取ることを考える. \\
  \begin{multicols}{2}
    $S$: \\
    \begin{center}
      \begin{tikzpicture}[aut]
        %node
        \node[state,initial,accepting]   (q0) {$q_0$};
        %output function
        \node[below = 1.7 of q0] {$F(q_0) = yx$};
        %arrows
        \draw [->]
          (q0) edge[loop above] node{
          $\top /
          \left[
            \begin{aligned}
              x &= x \cdot \id \\
              y &= \id \cdot y
            \end{aligned}
          \right]$
          } (q0)
        ;
      \end{tikzpicture}
    \end{center}

    \columnbreak

    $A$: \\
    \begin{center}
      \begin{tikzpicture}[aut]
        %nodes
        \node[state,initial] at (0, 0)    (p0) {$p_0$};
        \node[state,accepting] at (2, 1)  (p1) {$p_1$};
        \node[state] at (2, -1)           (p2) {$p_2$};
        %arrows
        \draw [->]
          (p0) edge[bend left]  node[above] {$a$}       (p1)
               edge[bend right] node[above] {$\lnot a$} (p2)
          (p1) edge[loop above] node {$\top$}           ()
          (p2) edge[loop below] node {$\top$}           ()
        ;
      \end{tikzpicture}
    \end{center}
  \end{multicols}

  $\forall c \in \Sigma.\ \id(c) = c, \seman{a} = \{ a \}$とする.

  以下のようなオートマトンが得られる.\\

  \begin{center}
    \begin{tikzpicture}[aut]
      %nodes
      \node[state,initial]    at (-1, 0)   (p) {$q'_0$};
      \node[state]            at (5, 4)   (q) {$(q_0, g_{\scriptsize \id})$};
      \node[state,accepting]  at (5, -4)  (r) {$(q_0, g_1)$};
      %arrows
      \draw [->]
        (q) edge               node {$a(\id)$}        (r)
            edge[loop above]   node {$\top$}             ()
        (p) edge[bend left]    node[above] {$\top$}      (q)
            edge[bend right]   node[below] {$a(\id)$} (r)
      ;
    \end{tikzpicture}
  \end{center}

  $q'_0$は新しく用意された開始状態で, 実質的には2状態である.

  $g_i \subseteq P \times X \rightarrow P $は以下の様な関数.
  \begin{table}[H]
    \begin{center}  \label{tables: g_i}
      \begin{tabular}{ccc}
        \begin{minipage}[t]{0.5\hsize}
          \begin{tabular}{c|c}
            $g_{\scriptsize \id}$ & $ $ \\
            \hline
            $x$ & $\{ (p_1, p_1) \}$ \\
            $y$ & $\{ (p_1, p_1) \}$ \\
          \end{tabular}
        \end{minipage} &

        \begin{minipage}[t]{0.5\hsize}
          \begin{tabular}{c|c}
            $g_1$ & $ $ \\
            \hline
            $x$ & $\{ (p_1, p_1) \}$ \\
            $y$ & $\{ (p_0, p_1) \}$ \\
          \end{tabular}
        \end{minipage} \\
      \end{tabular}
    \end{center}
  \end{table}

  ここで$g_{\scriptsize \id}$, $g_1$により$A$がどのように拡張されているかを表す.

  $\id$では実質的に変数がループであるとみなせる.
  \begin{center}
    \begin{tikzpicture}[aut]
      %nodes
      \node[state,initial] at (-2, 0)   (p0) {$p_0$};
      \node[state,accepting] at (2, 2)  (p1) {$p_1$};
      \node[state] at (2, -2)           (p2) {$p_2$};
      %arrows
      \draw [->]
        (p0) edge[bend left] node[above] {$a$} (p1)
             edge[bend right] node[above] {$\lnot a$} (p2)
        (p1) edge[loop above] node {$\top$} ()
        (p1) edge[loop right] node {$x, y$} ()
        (p2) edge[loop below] node {$\top$} ()
      ;
    \end{tikzpicture}
  \end{center}

  $g_1$では変数の中の文字列を読んでいくときの遷移で以下のようになるSFAを表すとみなせる.
  \begin{center}
    \begin{tikzpicture}[aut]
      %nodes
      \node[state,initial] at (-2, 0)   (p0) {$p_0$};
      \node[state,accepting] at (2, 2)  (p1) {$p_1$};
      \node[state] at (2, -2)           (p2) {$p_2$};
      %arrows
      \draw [->]
        (p0) edge[bend left] node[above] {$a$} (p1)
             edge[left] node[below] {$y$} (p1)
             edge[bend right] node[above] {$\lnot a$} (p2)
        (p1) edge[loop above] node {$\top$} ()
             edge[loop right] node {$x$} ()
        (p2) edge[loop below] node {$\top$} ()
      ;
    \end{tikzpicture}
  \end{center}
  ただし, 変数による遷移と論理式による遷移を区別するため別の矢印により表している.

  \section{上記構成の正しさ}

  \subsection{$\mathcal{L}(A') \supset S^{-1}(\mathcal{L}(A))$の証明}

  \begin{lemma} \label{relation_over_states_on_g_is_morphism}
    ある$p_2 \in P$に対して
    \[
      p_1 \transarrow{h_1 / \varphi_1}{g} p_2 \transarrow{h_2 / \varphi_2}{g} p_3
      \Leftrightarrow
        p_1 \transarrow{h_1 h_2 / \varphi_1 \land \varphi_2}{g} p_3
    \]
  \end{lemma}

  \begin{proof}
    $|h_1|$の帰納法で示す.
    $(\Rightarrow)$もほとんど同様に示すことができるため
    $(\Leftarrow)$のみ示す. \\
    $|h_1| = 0$のとき
    $p_1 \transarrow{\varepsilon h_2 / [\top] \land \varphi_2}{g} p_3$かつ
    $\top \land \varphi_2 = \varphi_2$より
    $p_1 \transarrow{\varepsilon / [\top]}{g} p_1 \transarrow{h_2 / \varphi_2}{g'} p_3$から成り立つ.

    $|h_1| = k$で成り立つと仮定して$|h_1| = k + 1$で成り立つことを示す.

    $h_1 = (\lambda, i) h, (\lambda, i) \in \Lambda^{[m]}$,
    $p_1 \transarrow{\varphi'_1}{A} p'_1 \in \delta$とする.
    $p_1 \transarrow{h_1 h_2 / \varphi_1 \land \varphi_2}{g} p_3$より\ref{rule: 1}から
    $[\varphi'_1(\lambda)]_i \land \varphi''_1 = \varphi_1$なる$\varphi''_1$に対して
    $p'_1 \transarrow{h h_2 / \varphi''_1 \land \varphi_2}{g} p_3$でなければならない.
    (仮定より$p_1 \transarrow{\varphi'_1}{A} p'_1$のようなパスが少なくとも一つある.)
    帰納法の仮定よりある$p_2 \in P$に対して
    \[
      p'_1 \transarrow{h / \varphi''_1}{g} p_2 \transarrow{h_2 / \varphi_2}{g} p_3
    \]
    より$p_1 \transarrow{h_1 / \varphi_1}{g} p_2$であることが分かるから示された.

    $h_1 = x h$のとき
    $p_1 \transarrow{h_1 h_2 / \varphi_1 \land \varphi_2}{g} p_3$より\ref{rule: 2}から
    $(p_1, p'_1) \in g(x)$なる$p'_1$に対して
    $p'_1 \transarrow{h h_2 / \varphi_1 \land \varphi_2}{g} p_3$でなければならない.
    帰納法の仮定よりある$p_2 \in P$に対して
    \[
      p'_1 \transarrow{h / \varphi_1}{g} p_2 \transarrow{h_2 / \varphi_2}{g} p_3
    \]
    より$p_1 \transarrow{h_1 / \varphi_1}{g} p_2$であることが分かるから示された.
  \end{proof}

  \begin{lemma} \label{relation_over_g_is_associative_right}
    $\alpha \in X \rightarrow (X \cup \Lambda)^*, \beta \in X \rightarrow (X \cup \Lambda^{[m]})^*$
    とする.
    $g \transarrow{\alpha / \varphi_1}{} g' \transarrow{\beta / \varphi_2}{} g''$であるとき,
    \[
      g \transarrow{\alpha \circ \beta / \varphi_1 \land \varphi_2^\uparrow}{} g''
    \]
    が成り立つ.
  \end{lemma}

  \begin{proof}
    $\varphi = \bigwedge_{x \in \scriptsize \dom(g'')} \bigwedge_{(p, p') \in g''(x)}
      \Delta_g(p, \alpha \circ \beta (x), p')$
    を計算する.
    $\beta(x)$に$k$個の変数が含まれるとして,
    $l_0y_0 \ldots l_{k - 1}y_{k - 1}l_k$としても一般性を失わない.($1 \leq k, l_i \in \Lambda^{[m]^*}$)
    以下では$r_0 = p, r'_k = p'$とする.
    \begin{align*}
      \varphi &= \bigwedge_{x \in \scriptsize \dom(g'')} \bigwedge_{(p, p') \in g''(x)}
                  \Delta_g(p, \alpha \circ \beta (x), p')  \\
              &= \bigwedge_{x \in \dom(g'')} \bigwedge_{(p, p') \in g''(x)}
                  \bigvee \left\{
                  \varphi_{l_0} \land \varphi_{y_0} \land \cdots
                  \land \varphi_{l_{k - 1}} \land \varphi_{y_{k - 1}} \land \varphi_{l_k}
                  \mid
                    \begin{aligned}
                      r_0 &\transarrow{\alpha(l_0) / \varphi_{l_0}}{g} r'_0
                        \transarrow{\alpha(y_0) / \varphi_{y_0}}{g} r_1  \\
                        &\transarrow{\alpha(l_1) / \varphi_{l_1}}{g} r'_1 \cdots \\
                        r_{k - 1} &\transarrow{\alpha(l_{k - 1}) / \varphi_{l_{k - 1}}}{g} r'_{k - 1}
                        \transarrow{\alpha(y_{k - 1}) / \varphi_{y_{k - 1}}}{g} r_k \\
                        &\transarrow{\alpha(l_k) / \varphi_{l_k}}{g} r'_k
                    \end{aligned}
                  \right\} \\
              &= \bigwedge_{x \in \scriptsize \dom(g'')} \bigwedge_{(p, p') \in g''(x)}
                  (
                  (\bigwedge_{i \in [0, \ldots, k]} \Delta_g(r_i, \alpha(l_i), r'_i)) \land
                  (\bigwedge_{i \in [0, \ldots k - 1]} \Delta_g(r'_i, \alpha(y_i), r_{i + 1}))
                  )  \\
              &= \bigwedge_{x \in \scriptsize \dom(g'')} \bigwedge_{(p, p') \in g''(x)}
                  (
                  (\bigwedge_{i \in [0, \ldots, k]} \Delta_{g'}(r_i, l_i, r'_i))^\uparrow \land
                  (\bigwedge_{i \in [0, \ldots k - 1]} \Delta_g(r'_i, \alpha(y_i), r_{i + 1}))
                  )  \\
              &= \bigwedge_{x \in \scriptsize \dom(g'')} \bigwedge_{(p, p') \in g''(x)}
                  (
                    \Delta_{g'}(p, \beta(x), p')^\uparrow \land
                    (\bigwedge_{i \in [0, \ldots, k - 1]} \Delta_g(r'_i, \alpha(y_i), r_{i + 1})
                  ) \\
              &= \varphi_2^\uparrow \land
                  \bigwedge_{x \in \scriptsize \dom(g'')} \bigwedge_{(p, p') \in g''(x)}
                    \bigwedge_{i \in [0, \ldots, k - 1]} \Delta_g(r'_i, \alpha(y_i), r_{i + 1})
    \end{align*}

    ここで, $\dom(g') = \bigcup_{y \in \scriptsize \dom(g'')} Var(\beta(y))$に注意すると,
    \begin{align*}
      \varphi_1 &= \bigwedge_{x \in \scriptsize \dom(g')} \bigwedge_{(p, p') \in g'(x)}
                  \Delta_g(p, \alpha(x), p')  \\
                &= \bigwedge_{x \in \scriptsize \dom(g'')} \bigwedge_{y \in Var(\beta(x))}
                  \bigwedge_{(p, p') \in g'(y)} \Delta_g(p, \alpha(y), p')  \\
                &= \bigwedge_{x \in \scriptsize \dom(g'')} \bigwedge_{y \in Var(\beta(x))}
                  \bigwedge_{(p, p') \in g'(y)} \bigvee_{p \transarrow{\alpha(y) / \varphi}{g} p'}
                  \varphi  \\
                &= \bigwedge_{x \in \scriptsize \dom(g'')} \bigwedge_{(p, p') \in g''(x)}
                  \bigwedge_{i \in [0, \ldots, k - 1]} \Delta_g(p'_i, \alpha(y_i), p_{i + 1})
    \end{align*}
    故に, $\varphi_1 \land \varphi_2^\uparrow = \varphi$が成り立つ.
  \end{proof}

  \begin{lemma} \label{relation_over_g_is_associative_left}
    $\alpha \in X \rightarrow (X \cup \Lambda)^*, \beta \in X \rightarrow (X \cup \Lambda^{[m]})^*$
    とする.
    $g \transarrow{\alpha \circ \beta / \varphi}{} g''$であるとき,
    ある$g'$が存在して,
    \[
      g \transarrow{\alpha / \varphi_1}{} g' \transarrow{\beta / \varphi_2}{} g'' \land
      \varphi = \varphi_1 \land \varphi_2^\uparrow
    \]
    が成り立つ.
  \end{lemma}

  \begin{proof}
    $x \in \bigcup_{y \in \scriptsize \dom(g'')} Var(\beta(y))$に対して
    \[
    g'(x) = \{ (r, r') \mid \exists y \in \scriptsize \dom(g'').\ \exists (p, p') \in g''(y).\
    \beta(y) = h_1 x h_2 \land p \transarrow{\alpha(h_1) / \varphi_1}{g} r
    \transarrow{\alpha(x) / \varphi}{g} r' \transarrow{\alpha(h_2) / \varphi_2}{g} p'\}
    \]
    として$g'$を定義する.
    定義より, $\dom(g') = \bigcup_{y \in \scriptsize \dom(g'')} Var(\beta(y))$であり
    $\varphi_2 = \bigwedge_{x \in \scriptsize \dom(g'')} \bigwedge_{(p, p') \in g''(x)}
    \Delta_{g'}(p, \beta(x), p')$として
    \[
      g' \transarrow{\beta / \varphi_2}{} g''
    \]
    が言える.

    \begin{align*}
      \dom(g) &= \bigcup_{y \in \scriptsize \dom(g'')} Var(\alpha \circ \beta(y)) \\
             &= \bigcup_{y \in \scriptsize \dom(g'')} (\bigcup_{z \in Var(\beta(y))} Var(\alpha(z))) \\
             &= \bigcup_{z \in \scriptsize \bigcup_{y \in \dom(g'')} Var(\beta(y))} Var(\alpha(z)) \\
             &= \bigcup_{z \in \scriptsize \dom(g')} Var(\alpha(z))
    \end{align*}
    より$\dom(g) = \bigcup_{y \in \scriptsize \dom(g')} Var(\alpha(y))$であり,
    $\varphi_1 = \bigwedge_{x \in \scriptsize \dom(g')} \bigwedge_{(p, p') \in g'(x)}
    \Delta_{g'}(p, \alpha(x), p')$として
    \[
      g \transarrow{\alpha / \varphi_1}{} g'
    \]
    が言える.

    \ref{relation_over_g_is_associative_right}と同様の変形により,
    \[
      \varphi = \varphi_1 \land \varphi_2^\uparrow
    \]
    である.
  \end{proof}

  \begin{lemma} \label{S_transition_and_g_tansition_are_derived_from_A'_transition}
    $(q, g) \transarrow{w}{} (q', g')$のとき,
    ある$\alpha, \psi, \phi$で以下が成り立つ.
    \begin{align*}
      q \transarrow{\psi / \alpha}{} q' \\
      g \transarrow{\alpha / \phi}{} g' \\
      \psi(w) \land \phi(w)
    \end{align*}
  \end{lemma}

  \begin{proof}
    $|w|$に関する帰納法で示す.
    $|w| = 0$のとき, $(q, g) \transarrow{\varepsilon}{} (q, g)$である.
    $q \transarrow{\top / \scriptsize \id}{} q$,
    $g \transarrow{\scriptsize \id / \top}{} g$より満たす.

    $|w| = k$で成り立つとき, $w = \sigma w'$として$|w| = k + 1$を考える.
    $(q, g) \transarrow{\sigma}{} (q'', g'') \transarrow{w'}{} (q', g')$となる.
    $(q, g) \transarrow{\sigma / \alpha}{} (q'', g'')$より,
    $\eta(\sigma)$なる$\eta$で$(q, g) \transarrow{\eta}{} (q'', g'')$が成り立つ.
    構成法より,
    \begin{align*}
      q \transarrow{\psi_1 / \alpha}{} q'' \\
      g \transarrow{\alpha / \phi_1}{} g'' \\
      \eta = \psi_1 \land \phi_1
    \end{align*}
    が成り立つ.
    $(q'', g'') \transarrow{w'}{} (q', g')$より帰納法の仮定から,
    \begin{align*}
      q'' \transarrow{\psi_2 / \alpha}{} q' \\
      g'' \transarrow{\alpha / \phi_2}{} g' \\
      \psi_2(w) \land \phi_2(w)
    \end{align*}
    \ref{relation_over_g_is_associative_right}より,
    \begin{align*}
      q \transarrow{\psi_1 \land \psi_2^\uparrow / \alpha}{} q' \\
      g \transarrow{\alpha / \phi_1 \land \phi_2^\uparrow}{} g'
    \end{align*}
    を満たす.
    $\psi_1 \land \psi_2^\uparrow = \psi, \phi_1 \land \phi_2^\uparrow = \phi$と置くと,
    $\psi(w)$かつ$\phi(w)$を満たすから$\psi(w) \land \phi(w)$より示された.
  \end{proof}

  \begin{lemma} \label{outputfunction_is_reducible}
    $w \in \mathcal{D}^*, v \in (X \cup \mathcal{D})^*$に対して
    $g \transarrow{\alpha / \phi}{} g'$, $p \transarrow{v}{g'} p'$, $\phi(w)$ならば
    \[
      p \transarrow{\alpha_w(v)}{g} p'
    \]
  \end{lemma}

  \begin{proof}
    $|v|$の帰納法で示す.
    $|v| = 0$ならば$p \transarrow{\varepsilon}{g'} p$であり,
    $\alpha(\varepsilon) = \varepsilon$より成り立つ.
    $|v| = k$で成り立つとして, $|v| = k + 1$を考える.

    $v = \sigma v', \sigma \in \mathcal{D}$を考える.
    $p \transarrow{\sigma}{} p'' \transarrow{v'}{} p'$とすると,
    帰納法の仮定より, $p'' \transarrow{\alpha_w(v')}{g} p'$である.
    $\alpha(\sigma) = \sigma$, $p \transarrow{\sigma}{g} p''$より成り立つ.
    $v = x v', x \in X$を考える.
    $(p, p'') \in g'(x)$で$p'' \transarrow{v'}{g'} p'$が成り立つ.
    帰納法の仮定より, $p'' \transarrow{\alpha_w(v')}{g} p'$である.
    ここで, $g \transarrow{\alpha / \phi}{} g'$と$\phi(w)$よりある$\varphi$で
    $p \transarrow{\alpha(x) / \varphi}{g} p''$かつ$\varphi(w)$が成り立つ.
    このとき$p \transarrow{\alpha_w(x)}{g} p''$である.

    ($\because$
    構成法より$\varphi$は$\alpha(x)$に文字列を適用したときに満たさなければならない論理式を全て含む.)
    よって$p \transarrow{\alpha_w(v)}{g} p'$.
  \end{proof}

  \begin{theorem} \label{symbolic_soundness}
    $L(A') \subseteq S^{-1}(L(A))$
  \end{theorem}

  \begin{proof}
    $g_0 \in ID, (q_0, g_0) \in I$とする.
    $(q_0, g_0) \transarrow{w}{} (q_f, g_f) \land (q_f, g_f) \in F'$であるとき,
    ある$\alpha, p_f \in P$で$q_0 \transarrow{w / \alpha_w}{S} q_f$,
    $p_0 \transarrow{\alpha_w(F_S(q_f))}{A} p_f \land p_f \in F$を満たすことを言えば良い.
    $(q_0, g_0) \transarrow{w}{} (q_f, g_f) \land (q_f, g_f) \in F'$であるとき,
    \ref{S_transition_and_g_tansition_are_derived_from_A'_transition}より
    ある$\alpha, \psi, \phi$で
    \begin{align*}
      q_0 \transarrow{\psi / \alpha}{S} q_f \\
      g_0 \transarrow{\alpha / \phi}{} g_f \\
      \psi(w) \land \phi(w)
    \end{align*}
    が成り立つ.
    故に, $q_0 \transarrow{w / \alpha_w}{S} q_f$が成り立つ.
    $(q_f, g_f) \in F'$より,
    $p_0 \transarrow{F_S(q_f)}{g_f} p_f \land p_f \in F$である.
    \ref{outputfunction_is_reducible}より
    $p_0 \transarrow{\alpha_w(F_S(q_f))}{\scriptsize \id} p_f$であり,
    $g_0$による遷移は$A$での遷移と一致するから示せた.
  \end{proof}

  \begin{lemma} \label{outputfunction_is_derivable}
    $h \in (X \cup \Lambda^{[m]})^*, w \in \mathcal{D}^*$とする.
    ある$\varphi$に対して
    \[
      p \transarrow{h_w}{g} p' \Rightarrow p \transarrow{h / \varphi}{g} p' \land \varphi(w)
    \]
  \end{lemma}

  \begin{proof}
    $|h|$に関する帰納法で示す.
    $|h| = 0$のとき$p \transarrow{\varepsilon}{g} p$かつ$p \transarrow{\varepsilon / \top}{g} p$より
    成り立つ.

    $|h| = k$で成り立つとして$k + 1$を考える.

    $h = (\lambda, i) h', \lambda \in \Lambda, 0 \leq i \le m$のとき,
    $a_i$を$w$の$i$文字目とすると
    $p \transarrow{\lambda(a_i)}{g} p'' \transarrow{h'_w}{g} p'$である.
    $p \transarrow{\lambda(a_i)}{g} p''$より
    ある$\varphi_1$で$p \transarrow{\varphi_1}{g} p''$かつ$\varphi_1(\lambda(a_i))$が成り立つ.
    構成法より$p \transarrow{(\lambda, i) / [\varphi_1(\lambda)]_i}{g} p''$.
    また, 帰納法の仮定より
    \[
      p'' \transarrow{h' / \varphi_2}{g} p' \land \varphi_2(w)
    \]
    である.
    \ref{relation_over_states_on_g_is_morphism}より,
    $p \transarrow{h / \varphi_1 \land \varphi_2}{g} p'$かつ
    $\varphi_1(w) \land \varphi_2(w)$より成り立つ.

    $h = x h'$のとき$(p, p'') \in g(x) \land p'' \transarrow{h_w}{g} p'$である.
    帰納法の仮定より
    \[
      p'' \transarrow{h' / \varphi}{g} p' \land \varphi(w)
    \]
    $p \transarrow{x / \top}{g} p''$であるから,
    \ref{relation_over_states_on_g_is_morphism}より,
    $p \transarrow{h / \varphi}{g} p''$であり成り立つ.
  \end{proof}

  \begin{theorem} \label{symbolic_completeness}
    $S^{-1}(L(A)) \subseteq L(A')$
  \end{theorem}

  \begin{proof}
    $q_0 \transarrow{w / \alpha_w}{S} q_f, q_f \in \dom(f)$かつ
    $p_0 \transarrow{\hat{\varepsilon}(\alpha_w(F_S(q_f)))}{A} p_f, p_f \in F$のとき
    ある$g_f, g_0$で
    \[
      (q_0, g_0) \transarrow{w}{A'} (q_f, g_f) \land (q_f, g_f) \in F' \land (q_0, g_0) \in I
    \]
    であることを示せば良い.

    $|w| = 0$のとき
    $q_0 \transarrow{\varepsilon / \scriptsize \id}{S} q_0, q_0 \in \dom(F_S)$かつ
    $p_0 \transarrow{\scriptsize \id(F_S(q_0)))}{A} p_f, p_f \in F$である.
    任意の$g_0 \in ID$に対して
    $(q_0, g_0) \transarrow{\varepsilon}{g_0} (q_0, g_0)$で,
    $g_0$により状態が変化しないことに注意すれば$p_0 \transarrow{F_S(q_0))}{g_0} p_f$より成り立つ.

    $|w| \ne 0$のとき,
    $q_0 \transarrow{w / \alpha_w}{S} q_f, q_f \in \dom(F_S)$より
    $q_0 \transarrow{\psi / \alpha}{S} q_f \land \psi(w)$なる遷移が存在する.

    $\alpha(F_S(q_f)) = w'_0z_0 \ldots w'_{l - 1}z_{l - 1}w'_l$として,
    $p_0 \transarrow{w'_0}{A} p_1 \transarrow{}{A} \cdots \transarrow{w_l}{} p_{l + 1}, p_{l + 1} \in F$
    とする.
    $g_0$を$z \in Var(\alpha_w(F_S(q_f)))$に対して
    $p_i \transarrow{w'_i}{A} p_{i + 1}$ならば$(p_{i + 1}, p_{i + 1}) \in g_0(z)$
    かつ, $(p_{i + 1}, p_{i + 1}) \in g_0(z)$ならば$\exists p_i.\ p_i \transarrow{w'_i}{A} p_{i + 1}$
    となるように取ると, $g_0 \in ID$かつ$\dom(g_0) = Var(\alpha_w(F_S(q_f)))$が成り立ち, $(q_0, g_0) \in ID$.

    また, $p_0 \transarrow{\hat{\varepsilon}(\alpha_w(F_S(q_f)))}{A} p_f, p_f \in F$だから
    $F_S(q_f)$中の変数が$k$個だとすれば
    $F_S(q_f) = w_0y_0 \ldots w_{k - 1}y_{k - 1}w_k, w_i \in \mathcal{D}^*, y_i \in X$として
    一般性を失わない. 故に遷移は$g_0$を用いて
    $p_0 \transarrow{w_0}{g_0} p'_0 \transarrow{\alpha_w(y_0)}{g_0}
    p_1 \transarrow{}{} \cdots \transarrow{\alpha_w(y_{k - 1})}{g_0}
    p_n \transarrow{w_k}{g_0} p_f$とかける.
    ここで, 上記$p_{l + 1}$と$p_f$は同一の状態である.

    $x \in \bigcup_{0 \leq i \le n} y_i$に対して
    $g_f(x) = \{ (p'_i, p_{i + 1}) \mid p'_i \transarrow{\alpha_w(x)}{g_0} p_{i + 1} \}$
    と$g_f$を定義すると
    $g_0 \transarrow{\alpha / \phi}{} g_f$.  \\
    また, $p_0 \transarrow{\alpha_w(F_S(q_f))}{\scriptsize \id} p_f$と$g_f$の定義より
    $p_0 \transarrow{F_S(q_f)}{g_f} p_f$が分かる.
    \ref{outputfunction_is_derivable}より
    $p'_i \transarrow{\alpha(y_i) / \varphi_i}{\scriptsize \id} p_{i + 1}$かつ$\varphi_i(w)$であるから,
    $\forall x \in \dom(g_f).\ \forall (p, p') \in g_f(x).\
    \Delta_{g_0}(p, \alpha(x), p')(w)$.
    故に, $\phi(w)$を満たす.
    以上より,
    $q_0 \transarrow{\psi / \alpha}{S} q_f, q_f \in \dom(F_S)$かつ
    $g_0 \transarrow{\alpha / \phi}{} g_f$である.
    $w = a_0a_1 \ldots a_n$とおけば,
    $q_0 \transarrow{\psi_0 / \alpha_0}{S} q_1 \transarrow{}{} \cdots
    \transarrow{}{} q_n \transarrow{\psi_n / \alpha_n}{S} q_f$とかけて,
    $\psi_i(a_i)$かつ$\alpha = \alpha_0 \circ \alpha_1 \circ \cdots \circ \alpha_n$が成り立つ.
    \ref{relation_over_g_is_associative_left}より
    $g_0 \transarrow{\alpha_0 / \phi_0}{} g_1 \transarrow{}{} \cdots
    \transarrow{}{} g_n \transarrow{\alpha_n / \phi_n}{} g_f$かつ
    $\phi = \phi_0 \land \phi_1 \land \cdots \land \phi_n^{\uparrow^{(n)}}$が成り立つ.
    故に構成法から
    $(q_0, g_0) \transarrow{\psi_0 \land \phi_0}{A'} (q_1, g_1) \transarrow{}{} \cdots
    \transarrow{}{} (q_n, g_n) \transarrow{\psi_n \land \phi_n}{A'} (q_f, g_f)$
    であり,
    \begin{align*}
      (\psi_0 \land \phi_0) \land (\psi_1 \land \phi_1)^\uparrow \land \cdots
      \land (\psi_n \land \phi_n)^{\uparrow^{(n)}} &=
      (\psi_0 \land \phi_0) \land (\psi_1^\uparrow \land \phi_1^\uparrow) \land \cdots
      \land (\psi_n^{\uparrow^{(n)}} \land \phi_n^{\uparrow^{(n)}}) \\
      &= (\psi_0 \land \psi_1^\uparrow \land \cdots \psi_n^{\uparrow^{(n)}}) \land
      (\phi_0 \land \phi_1^\uparrow \land \cdots \phi_n^{\uparrow^{(n)}}) \\
      &= \psi \land \phi
    \end{align*}
    より
    $(q_0, g_0) \transarrow{\psi \land \phi}{A'} (q_f, g_f)$
    が言える.
    $\psi(w) \land \phi(w)$であるから
    $(q_0, g_0) \transarrow{w}{A'} (q_f, g_f)$
    で, $q_f \in \dom(F_S)$かつ$p_0 \transarrow{F_S(q_f)}{g_f} p_f, p_f \in F$
    より$(q_f, g_f) \in F'$.
  \end{proof}

  \ref{symbolic_soundness}と\ref{symbolic_completeness}より直ちに以下が分かる.

  \begin{theorem}
    $L(A') = S^{-1}(L(A))$
  \end{theorem}

  よって, 上記構成の正当性が示せた.

\end{document}
