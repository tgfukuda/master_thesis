\documentclass[uplatex,dvipdfmx,a4j]{jsreport}
\usepackage{thesis}

\begin{document}
  \chapter{結論}  \label{chap:conclusion}

  本研究では Symbolic SSTの形式化とそれを利用した文字列制約の充足可能性判定を行った.
  Symbolic SSTにおいて遷移を論理式にすることは出力に入力の情報が直接関係してくるということであり,
  文字のインデックスの情報を加えた論理式を使った意味論を与えることでSymbolic SSTの意味を与えた.
  また, その逆像を与えるアルゴリズムを提案し, その正当性を示した.
  本論文のアルゴリズムはSymbolic SSTに関して制約を与えておらず, Copyful SSTにおいても逆像計算が可能である.
  これは状態として持たせた, 後段のSFAを変数に対して拡張する関数で冪集合を用いていることが要因であり,
  部分関数にする等の手法で逆像計算の定義域をcopyless(bounded copy)なものに制限することができるだろう.
  本研究において文字列制約は整数制約を含まないものを考えているが, これは逆像の手法に関係しており重要である.

  まず, 整数制約を含む文字列制約に対して拡張するためにはCEFAやParikh Automataのような整数に関する情報を
  オートマトンの遷移に持たせ, 半線形集合として表現しなければならないが, それらのモデルをSymbolicに拡張することが必要である.
  また, それらのモデルがSymbolic SSTの逆像で閉じていることが必要である.

  例えば, \cite{kamano2021solver}で用いている手法で, SSTによるNFAの逆像を計算する場合SSTはBounded copy
  であることが必要である.
  SSTの各変数が各ステップで, どのようなベクトルで更新されるか, どの変数のベクトルが利用されるかを行列で表し,
  それを状態として持たせることでAffine Parikh Automataを構成, それをParikh Automataに変換することで.
  計算を行っている.
  すなわち, この行列の種類が有限になるような場合でなければAffine Parikh AutomataからParikh Automataへの計算が
  できない.
  これが可能になる条件がBounded copyであるということになるが, 本論文の手法そのままでは閉じた写像が得られない.

  加えて, \cite{chen2020decision}, \cite{kamano2021solver}では整数制約を含む文字列制約に対して計算できること
  が共通点として挙げられるが, これらは文字列の等号否定も整数制約に帰着することで解くことができる.
  文字列の等号否定は
  \begin{itemize}
    \item 文字列の長さが異なる
    \item 文字列の$p$番目の文字が異なる
  \end{itemize}
  のいずれかを満たすという条件に変換でき, 整数制約として表すことができる.
  しかし, 二番目の条件はCEFA, Parikh Automataで表現する場合アルファベットのサイズに依存した
  ベクトルが必要となる.
  そのため, 等号否定に関してこの手法をそのままSymbolicにすることはできない.

  こういった理由から本研究の手法を整数制約に対しても拡張する場合,
  何らかの制限を与えなければ適用することはできないと考えられ, またSymbolicでない場合に計算が可能だったクラス
  であっても計算が可能かわからないことなどが今後の課題の一つである.
\end{document}
