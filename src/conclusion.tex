\documentclass[uplatex,dvipdfmx,a4j]{jsreport}
\usepackage{thesis}

\begin{document}
  \chapter{結論}  \label{chap:conclusion}

  本研究では Symbolic SSTの形式化とそれを利用した文字列制約の充足可能性判定を行った.
  Symbolic SSTにおいて遷移を論理式にすることは出力に入力の情報が直接関係してくるということであり,
  文字のインデックスの情報を加えた論理式を使った意味論を与えることでSymbolic SSTの意味を与えた.
  また, その逆像を与えるアルゴリズムを提案し, その正当性を示した.
  本論文のアルゴリズムはSymbolic SSTに関して制約を与えておらず, Copyful SSTにおいても同様の手法により
  逆像計算が行える.
  これは状態として持たせた, 後段のSFAを変数に対して拡張する関数で冪集合を用いていることが要因であり,
  部分関数にする等の手法で逆像計算の定義域をcopyless(bounded copy)なものに制限することができるだろう.
  本研究において文字列制約は整数制約を含まないものを考えているが, これは逆像の手法に関係しており重要である.

  まず, 整数制約を含む文字列制約に対して拡張するためにはCEFAやParikh Automataのような整数に関する情報を
  オートマトンの遷移に持たせ, 半線形集合として表現しなければならないが,
  それらのモデルをSymbolicに拡張することが必要である.
  また, それらのモデルがSymbolic SSTの逆像で閉じていることが必要である.
  こういった理由から, 本研究の手法に何らかの修正を与えなければ適用することはできないと考えられ, 今後の課題の一つである.

  (計算量周り, ボトルネックの考察入れる?)

  実装面に関しても, 非常に簡単な制約に関しては動作の確認が取れているものの, うまく動かないケースがあるため
  改善が必要である.
\end{document}
