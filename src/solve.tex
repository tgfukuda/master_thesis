\documentclass[uplatex,dvipdfmx,a4j]{jsreport}
\usepackage{thesis}

\begin{document}
  \chapter{文字列制約の充足可能性判定} \label{chap:solve}

  文字列制約$\varphi$の充足可能性判定を以下のように行う.
  ただし, $\varphi_{sl} \defeq \bigwedge_{m \leq i \le n} \varphi_i$であるとする.
  \begin{enumerate}
    \item 直線制約をSymbolic SSTとして, 正規制約をSFAとして表現する.  \label{step_1}
    \item SFAのSSTによる逆像を順に取る.                            \label{step_2}
    \item 最終的に生成されたSFAの空性を判定することで解を得る.         \label{step_3}
  \end{enumerate}
  {\bf step\ref{step_1}}

  Zhuによるソルバと同様にして変換を行う.
  $x_i$に対応して得られるSST$S_i$はセパレーター$\#$として
  入力$w_0 \# \ldots \# w_{i - 1} \#$に対して$w_0 \# \ldots \# w_i \#$を出力するもので,
  $w_i$が$x_i$の制約を満たすものとなるように構成される.
  \begin{example}
    $x_2 = \mbox{reverse}(x_0)$に対して出力されるSSTは以下のようになる.
    $\varphi_{sep}(\sigma) = (\sigma = \#)$とする.
    \begin{center}
      \begin{tikzpicture}[aut]
        %nodes
        \node[state,initial]    at (-5, 0) (p0) {$p_0$};
        \node[state]            at (0, 0)  (p1) {$p_1$};
        \node[state,accepting]  at (5, 0)  (p2) {$p_2$};
        %output function
        \node[right=2 of p2] {$F(p_2) = x r \#$};
        %arrows
        \draw [->]
          (p0)
            edge[loop above] node{
              $\lnot \varphi_{sep} /
              \left[
                \begin{aligned}
                  x &= x \cdot \id \\
                  r &= \id \cdot r
                \end{aligned}
              \right]$
            } ()
            edge[right] node[below=0.3] {
               $\varphi_{sep} /
               \left[
                 \begin{aligned}
                   x &= x \cdot \id \\
                   r &= r
                 \end{aligned}
               \right]$
            } (p1)
          (p1)
            edge[loop above] node{
              $\lnot \varphi_{sep} /
              \left[
                \begin{aligned}
                  x &= x \cdot \id \\
                  r &= r
                \end{aligned}
              \right]$
            } ()
            edge[right] node[below=0.3] {
              $\varphi_{sep} /
              \left[
                \begin{aligned}
                  x &= x \cdot \id \\
                  r &= r
                \end{aligned}
              \right]$
            } (p2)
        ;
      \end{tikzpicture}
    \end{center}
  \end{example}
  ただし, 正規制約の情報をSSTに入れることはせず, $x_i$の制約$\varphi_i$の計算に必要な情報を用いて
  SSTの構成を行う.

  正規制約$\bigwedge_{0 \leq i \le n} x_i \in R_i$より
  $R_0 \# R_1 \# \cdots \# R_{n - 1} \#$に対応するSFAを構成する.
  対応する正規制約のない$x_i$については$x_i \in \mathcal{D}^*$であるものとして考える.
  このSFAは正規制約を満たす$w_0, \ldots, w_{n - 1}$をセパレーターで連接した文字列を受理することに注意する.

  これによって直線制約が$\bigwedge_{m \leq i \le n} \varphi_i$であるとき
  $n - m$個のSSTと1個のSFAが得られる. \\
  {\bf step\ref{step_2}}

  step\ref{step_1}で生成された$x_i$に対応するSSTは$x_0, \ldots, x_{i - 1}$までを入力として受け取り,
  それらにより得られる$x_i$を末尾に追加して出力する.
  $S_{n - 1}$によってSFAの逆像を取ると, 得られたSFAは正規制約を満たす
  $w_0, \ldots, w_{n - 2}$をセパレーターで連接した文字列を受理する.
  同様にして得られた, 正規制約満たす$w_0, \ldots, w_i$をセパレーターで連接した文字列を受理するSFAを順に
  $S_i$で逆像を取る.  \\
  {\bf step\ref{step_3}}

  step\ref{step_2}で最終的に$0 < m$であれば正規制約満たす$w_0 \# \ldots \# w_{m - 1}$を,
  $m = 0$であれば$\varepsilon$を受理するSFAか, $\emptyset$のSFAが得られる.
  得られたSFAの空性を調べることで充足可能性を判定することができ, 空でない場合は
  受理されるパスを取ってくることで解の一つを得ることができる.
\end{document}
