\documentclass[uplatex,dvipdfmx,a4j]{jsreport}
\usepackage{thesis}

\begin{document}
  \chapter{関連研究}  \label{chap:related_works}

  本研究の重要な先行研究としてZhuらによる\cite{zhu2019sstsolver}がある.
  ZhuらはまずSSTが合成で閉じていることを利用して, 本研究のソルバで用いているSSTに似た形式に変換し, 逆像ではなく
  合成を用いて基礎直線制約と正規制約を表すSSTを作った.
  そしてそのソルバを出力のParikh Imageを表すトランスデューサーに変換し, その半線形集合の空性を調べることにより
  充足可能性を判定している.
  ここで用いているSSTはSymbolicなモデルではなく, 決定性である.

  釜野らによる\cite{kamano2021solver}ではParikh Automataと非決定性SSTを組み合わせた
  Parikh SSTを用いて文字列の長さに関する整数制約や文字と整数により表現されるトランスダクション(indexOfやsubStr等)
  を含む文字列制約に対応した. Parikh Automataは遷移にベクトルを付加し, 整数の半線形集合を評価後の値が満たすかどうか
  調べることができるモデルで, 通常のオートマトンより文字列のextended Prikh Imageを用いた制限をつけることができる分
  高い表現力を持つ. Chenらによるソルバとは手法が異なるが, CEFAと本質的には近いものである.
  釜野らはParikh SSTが合成に関して閉じていること, 逆像の計算が可能であることを利用してそれぞれの手法により実装を行っている.

  Chenらによる\cite{chen2020decision}では, 本研究で構成している逆像とはかなり異なる方法だが,
  CEFAとトランスデューサーを用いて逆像の計算を行い, CEFAの到達可能性に帰着している.
  本研究では, 整数制約を含まない点と直線制約に含まれるトランスダクションがSSTによるものでも構わないという点で異なる.
  また, 同Chenらは\cite{chen2022solving}においてもPSST(Prioritized Streaming String Transducers)
  を用いた逆像によるソルバを提案しており, このソルバではそれまでOstrichを
  キャプチャグループ付きの正規表現を含む文字列制約に対しても拡張している.
\end{document}
